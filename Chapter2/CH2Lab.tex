% Options for packages loaded elsewhere
\PassOptionsToPackage{unicode}{hyperref}
\PassOptionsToPackage{hyphens}{url}
%
\documentclass[
]{article}
\usepackage{lmodern}
\usepackage{amssymb,amsmath}
\usepackage{ifxetex,ifluatex}
\ifnum 0\ifxetex 1\fi\ifluatex 1\fi=0 % if pdftex
  \usepackage[T1]{fontenc}
  \usepackage[utf8]{inputenc}
  \usepackage{textcomp} % provide euro and other symbols
\else % if luatex or xetex
  \usepackage{unicode-math}
  \defaultfontfeatures{Scale=MatchLowercase}
  \defaultfontfeatures[\rmfamily]{Ligatures=TeX,Scale=1}
\fi
% Use upquote if available, for straight quotes in verbatim environments
\IfFileExists{upquote.sty}{\usepackage{upquote}}{}
\IfFileExists{microtype.sty}{% use microtype if available
  \usepackage[]{microtype}
  \UseMicrotypeSet[protrusion]{basicmath} % disable protrusion for tt fonts
}{}
\makeatletter
\@ifundefined{KOMAClassName}{% if non-KOMA class
  \IfFileExists{parskip.sty}{%
    \usepackage{parskip}
  }{% else
    \setlength{\parindent}{0pt}
    \setlength{\parskip}{6pt plus 2pt minus 1pt}}
}{% if KOMA class
  \KOMAoptions{parskip=half}}
\makeatother
\usepackage{xcolor}
\IfFileExists{xurl.sty}{\usepackage{xurl}}{} % add URL line breaks if available
\IfFileExists{bookmark.sty}{\usepackage{bookmark}}{\usepackage{hyperref}}
\hypersetup{
  hidelinks,
  pdfcreator={LaTeX via pandoc}}
\urlstyle{same} % disable monospaced font for URLs
\usepackage[margin=1in]{geometry}
\usepackage{color}
\usepackage{fancyvrb}
\newcommand{\VerbBar}{|}
\newcommand{\VERB}{\Verb[commandchars=\\\{\}]}
\DefineVerbatimEnvironment{Highlighting}{Verbatim}{commandchars=\\\{\}}
% Add ',fontsize=\small' for more characters per line
\usepackage{framed}
\definecolor{shadecolor}{RGB}{248,248,248}
\newenvironment{Shaded}{\begin{snugshade}}{\end{snugshade}}
\newcommand{\AlertTok}[1]{\textcolor[rgb]{0.94,0.16,0.16}{#1}}
\newcommand{\AnnotationTok}[1]{\textcolor[rgb]{0.56,0.35,0.01}{\textbf{\textit{#1}}}}
\newcommand{\AttributeTok}[1]{\textcolor[rgb]{0.77,0.63,0.00}{#1}}
\newcommand{\BaseNTok}[1]{\textcolor[rgb]{0.00,0.00,0.81}{#1}}
\newcommand{\BuiltInTok}[1]{#1}
\newcommand{\CharTok}[1]{\textcolor[rgb]{0.31,0.60,0.02}{#1}}
\newcommand{\CommentTok}[1]{\textcolor[rgb]{0.56,0.35,0.01}{\textit{#1}}}
\newcommand{\CommentVarTok}[1]{\textcolor[rgb]{0.56,0.35,0.01}{\textbf{\textit{#1}}}}
\newcommand{\ConstantTok}[1]{\textcolor[rgb]{0.00,0.00,0.00}{#1}}
\newcommand{\ControlFlowTok}[1]{\textcolor[rgb]{0.13,0.29,0.53}{\textbf{#1}}}
\newcommand{\DataTypeTok}[1]{\textcolor[rgb]{0.13,0.29,0.53}{#1}}
\newcommand{\DecValTok}[1]{\textcolor[rgb]{0.00,0.00,0.81}{#1}}
\newcommand{\DocumentationTok}[1]{\textcolor[rgb]{0.56,0.35,0.01}{\textbf{\textit{#1}}}}
\newcommand{\ErrorTok}[1]{\textcolor[rgb]{0.64,0.00,0.00}{\textbf{#1}}}
\newcommand{\ExtensionTok}[1]{#1}
\newcommand{\FloatTok}[1]{\textcolor[rgb]{0.00,0.00,0.81}{#1}}
\newcommand{\FunctionTok}[1]{\textcolor[rgb]{0.00,0.00,0.00}{#1}}
\newcommand{\ImportTok}[1]{#1}
\newcommand{\InformationTok}[1]{\textcolor[rgb]{0.56,0.35,0.01}{\textbf{\textit{#1}}}}
\newcommand{\KeywordTok}[1]{\textcolor[rgb]{0.13,0.29,0.53}{\textbf{#1}}}
\newcommand{\NormalTok}[1]{#1}
\newcommand{\OperatorTok}[1]{\textcolor[rgb]{0.81,0.36,0.00}{\textbf{#1}}}
\newcommand{\OtherTok}[1]{\textcolor[rgb]{0.56,0.35,0.01}{#1}}
\newcommand{\PreprocessorTok}[1]{\textcolor[rgb]{0.56,0.35,0.01}{\textit{#1}}}
\newcommand{\RegionMarkerTok}[1]{#1}
\newcommand{\SpecialCharTok}[1]{\textcolor[rgb]{0.00,0.00,0.00}{#1}}
\newcommand{\SpecialStringTok}[1]{\textcolor[rgb]{0.31,0.60,0.02}{#1}}
\newcommand{\StringTok}[1]{\textcolor[rgb]{0.31,0.60,0.02}{#1}}
\newcommand{\VariableTok}[1]{\textcolor[rgb]{0.00,0.00,0.00}{#1}}
\newcommand{\VerbatimStringTok}[1]{\textcolor[rgb]{0.31,0.60,0.02}{#1}}
\newcommand{\WarningTok}[1]{\textcolor[rgb]{0.56,0.35,0.01}{\textbf{\textit{#1}}}}
\usepackage{graphicx,grffile}
\makeatletter
\def\maxwidth{\ifdim\Gin@nat@width>\linewidth\linewidth\else\Gin@nat@width\fi}
\def\maxheight{\ifdim\Gin@nat@height>\textheight\textheight\else\Gin@nat@height\fi}
\makeatother
% Scale images if necessary, so that they will not overflow the page
% margins by default, and it is still possible to overwrite the defaults
% using explicit options in \includegraphics[width, height, ...]{}
\setkeys{Gin}{width=\maxwidth,height=\maxheight,keepaspectratio}
% Set default figure placement to htbp
\makeatletter
\def\fps@figure{htbp}
\makeatother
\setlength{\emergencystretch}{3em} % prevent overfull lines
\providecommand{\tightlist}{%
  \setlength{\itemsep}{0pt}\setlength{\parskip}{0pt}}
\setcounter{secnumdepth}{-\maxdimen} % remove section numbering

\author{}
\date{\vspace{-2.5em}}

\begin{document}

\begin{enumerate}
\def\labelenumi{\arabic{enumi}.}
\setcounter{enumi}{7}
\tightlist
\item
  This exercise relates to the \textbf{College} data set, which can be
  found in the file \textbf{College.csv.} It contains a number of
  variables for 777 different universities and colleges in the US. The
  variables are

  \begin{itemize}
  \tightlist
  \item
    \textbf{Private} : Public/private indicator
  \item
    \textbf{Apps} : Number of applications received
  \item
    \textbf{Accept} : Number of applicants accepted
  \item
    \textbf{Enroll} : Number of new students enrolled
  \item
    \textbf{Top10perc} : New students from the top 10 \% of high school
    class
  \item
    \textbf{top25perc} : New students from the top 25 \% of high school
    class
  \item
    \textbf{F.Undergrad} : Number of full-time undergraduates
  \item
    \textbf{P.Undergrad} : Number of part-time undergraduates
  \item
    \textbf{Outstate} : Out-of-state tuition
  \item
    \textbf{Room.Board} : Room and board costs
  \item
    \textbf{Books} : Estimated book costs
  \item
    \textbf{Personal} : Estimated personal spending
  \item
    \textbf{PhD} : Percent of faculty with Ph.D's
  \item
    \textbf{Terminal} : Percent of faculty with terminal degree
  \item
    \textbf{S.F.Ratio} : Student/faculty ratio
  \item
    \textbf{perc.alumni} : Percent of alumni who donate
  \item
    \textbf{Expend} : Instructional expenditure per student
  \item
    \textbf{Grad.Rate} : Graduation rate
  \end{itemize}

  Before reading the data into \textbf{R}, it can be viewed in Excel or
  a text editor.

  \begin{enumerate}
  \def\labelenumii{\alph{enumii})}
  \tightlist
  \item
    Use the \textbf{read.csv()} function to read the data into
    \textbf{R.} Call the loaded data \textbf{college.} Make sure that
    you have the directoy set to the correct location for the data.
  \end{enumerate}
\end{enumerate}

\begin{Shaded}
\begin{Highlighting}[]
\KeywordTok{library}\NormalTok{(ISLR)}
\end{Highlighting}
\end{Shaded}

\begin{verbatim}
  b) Look at the ata using the **fix()** function. You should notice that the first column is just the name of each university. We on't really want **R** to treat this as data. However, it may be handy to have these names for later. Try the following commands:
  
\end{verbatim}

\begin{Shaded}
\begin{Highlighting}[]
\CommentTok{#NOTE: ISLR PACKAGE ALREADY HAS THIS DATASET}
\CommentTok{#rownames(college)=college[,1]}
\CommentTok{#fix(college)}
\end{Highlighting}
\end{Shaded}

\begin{verbatim}
  You should see that there is now a **row.names** column with the name of each university recorded. This means that **R** has given each row a name corresponding to the appropriate university. **R** will not try to perform calculations on the row names. However, we still need to eliminate the first column in the data where the names are stored. Try
\end{verbatim}

\begin{Shaded}
\begin{Highlighting}[]
\CommentTok{#NOTE: ISLR PACKAGE ALREADY HAS THIS DATASET}
\CommentTok{#college=college[,1]}
\CommentTok{#fix(college)}
\end{Highlighting}
\end{Shaded}

\begin{verbatim}
  Now you should see that the first data column is **Private.** Note that another column labeled **row.names** now appears before the **Private** column. However, this is not a data column but rather the name that **R** is giving to each row.
  c) 
    i. Use the **summary()** function to produce a numerical summary of the variables in the data set.
\end{verbatim}

A list using small letters:

\begin{enumerate}
\def\labelenumi{\alph{enumi})}
\tightlist
\item
  Something
\item
  Something else
\end{enumerate}

\end{document}
